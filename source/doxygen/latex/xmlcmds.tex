Doxygen supports most of the X\+ML commands that are typically used in C\# code comments. The X\+ML tags are defined in Appendix D of the \href{https://www.ecma-international.org/publications-and-standards/standards/ecma-334/}{\texttt{ E\+C\+M\+A-\/334}} standard, which defines the C\# language. Unfortunately, the specification is not very precise and a number of the examples given are of poor quality.

Here is the list of tags supported by doxygen\+:

\tabulinesep=1mm
\begin{longtabu}spread 0pt [c]{*{2}{|X[-1]}|}
\hline
\cellcolor{\tableheadbgcolor}\textbf{ X\+ML Command}&\cellcolor{\tableheadbgcolor}\textbf{ Description }\\\cline{1-2}
\endfirsthead
\hline
\endfoot
\hline
\cellcolor{\tableheadbgcolor}\textbf{ X\+ML Command}&\cellcolor{\tableheadbgcolor}\textbf{ Description }\\\cline{1-2}
\endhead
{\ttfamily \label{xmlcmds_xmltag_c}%
\Hypertarget{xmlcmds_xmltag_c}%
\index{$<$c$>$@{$<$c$>$}} $<$c$>$}&Identifies inline text that should be rendered as a piece of code. Similar to using {\ttfamily $<$tt$>$}text{\ttfamily $<$/tt$>$}. \\\cline{1-2}
{\ttfamily \label{xmlcmds_xmltag_code}%
\Hypertarget{xmlcmds_xmltag_code}%
\index{$<$code$>$@{$<$code$>$}} $<$code$>$}&Set one or more lines of source code or program output. Note that this command behaves like \mbox{\hyperlink{commands_cmdcode}{\textbackslash{}code}} ... \mbox{\hyperlink{commands_cmdendcode}{\textbackslash{}endcode}} for C\# code, but it behaves like the H\+T\+ML equivalent \mbox{\hyperlink{htmlcmds_htmltag_CODE}{$<$C\+O\+DE$>$}}...\mbox{\hyperlink{htmlcmds_htmltag_endCODE}{$<$/\+C\+O\+DE$>$}} for other languages. \\\cline{1-2}
{\ttfamily \label{xmlcmds_xmltag_description}%
\Hypertarget{xmlcmds_xmltag_description}%
\index{$<$description$>$@{$<$description$>$}} $<$description$>$}&Part of a \mbox{\hyperlink{xmlcmds_xmltag_list}{$<$list$>$}} command, describes an item. \\\cline{1-2}
{\ttfamily \label{xmlcmds_xmltag_example}%
\Hypertarget{xmlcmds_xmltag_example}%
\index{$<$example$>$@{$<$example$>$}} $<$example$>$}&Marks a block of text as an example, ignored by doxygen. \\\cline{1-2}
{\ttfamily \label{xmlcmds_xmltag_exception}%
\Hypertarget{xmlcmds_xmltag_exception}%
\index{$<$exception$>$@{$<$exception$>$}} $<$exception cref=\char`\"{}member\char`\"{}$>$}&Identifies the exception a method can throw. \\\cline{1-2}
{\ttfamily \label{xmlcmds_xmltag_include}%
\Hypertarget{xmlcmds_xmltag_include}%
\index{$<$include$>$@{$<$include$>$}} $<$include$>$}&Can be used to import a piece of X\+ML from an external file. Ignored by doxygen at the moment. \\\cline{1-2}
{\ttfamily \label{xmlcmds_xmltag_inheritdoc}%
\Hypertarget{xmlcmds_xmltag_inheritdoc}%
\index{$<$inheritdoc$>$@{$<$inheritdoc$>$}} $<$inheritdoc$>$}&Can be used to insert the documentation of a member of a base class into the documentation of a member of a derived class that reimplements it. \\\cline{1-2}
{\ttfamily \label{xmlcmds_xmltag_item}%
\Hypertarget{xmlcmds_xmltag_item}%
\index{$<$item$>$@{$<$item$>$}} $<$item$>$}&List item. Can only be used inside a \mbox{\hyperlink{xmlcmds_xmltag_list}{$<$list$>$}} context. \\\cline{1-2}
{\ttfamily \label{xmlcmds_xmltag_list}%
\Hypertarget{xmlcmds_xmltag_list}%
\index{$<$list$>$@{$<$list$>$}} $<$list type=\char`\"{}type\char`\"{}$>$}&Starts a list, supported types are {\ttfamily bullet} or {\ttfamily number} and {\ttfamily table}. A list consists of a number of {\ttfamily $<$item$>$} tags. A list of type table, is a two column table which can have a header. \\\cline{1-2}
{\ttfamily \label{xmlcmds_xmltag_listheader}%
\Hypertarget{xmlcmds_xmltag_listheader}%
\index{$<$listheader$>$@{$<$listheader$>$}} $<$listheader$>$}&Starts the header of a list of type \char`\"{}table\char`\"{}. \\\cline{1-2}
{\ttfamily \label{xmlcmds_xmltag_para}%
\Hypertarget{xmlcmds_xmltag_para}%
\index{$<$para$>$@{$<$para$>$}} $<$para$>$}&Identifies a paragraph of text. \\\cline{1-2}
{\ttfamily \label{xmlcmds_xmltag_param}%
\Hypertarget{xmlcmds_xmltag_param}%
\index{$<$param$>$@{$<$param$>$}} $<$param name=\char`\"{}param\+Name\char`\"{}$>$}&Marks a piece of text as the documentation for parameter \char`\"{}param\+Name\char`\"{}. Similar to using \mbox{\hyperlink{commands_cmdparam}{\textbackslash{}param}}. \\\cline{1-2}
{\ttfamily \label{xmlcmds_xmltag_paramref}%
\Hypertarget{xmlcmds_xmltag_paramref}%
\index{$<$paramref$>$@{$<$paramref$>$}} $<$paramref name=\char`\"{}param\+Name\char`\"{}$>$}&Refers to a parameter with name \char`\"{}param\+Name\char`\"{}. Similar to using \mbox{\hyperlink{commands_cmda}{\textbackslash{}a}}. \\\cline{1-2}
{\ttfamily \label{xmlcmds_xmltag_permission}%
\Hypertarget{xmlcmds_xmltag_permission}%
\index{$<$permission$>$@{$<$permission$>$}} $<$permission$>$}&Identifies the security accessibility of a member. Ignored by doxygen. \\\cline{1-2}
{\ttfamily \label{xmlcmds_xmltag_remarks}%
\Hypertarget{xmlcmds_xmltag_remarks}%
\index{$<$remarks$>$@{$<$remarks$>$}} $<$remarks$>$}&Identifies the detailed description.  \\\cline{1-2}
{\ttfamily \label{xmlcmds_xmltag_returns}%
\Hypertarget{xmlcmds_xmltag_returns}%
\index{$<$returns$>$@{$<$returns$>$}} $<$returns$>$}&Marks a piece of text as the return value of a function or method. Similar to using \mbox{\hyperlink{commands_cmdreturn}{\textbackslash{}return}}. \\\cline{1-2}
{\ttfamily \label{xmlcmds_xmltag_see}%
\Hypertarget{xmlcmds_xmltag_see}%
\index{$<$see$>$@{$<$see$>$}} $<$see cref=\char`\"{}member\char`\"{}$>$}&Refers to a member. Similar to \mbox{\hyperlink{commands_cmdref}{\textbackslash{}ref}}. \\\cline{1-2}
{\ttfamily \label{xmlcmds_xmltag_seealso}%
\Hypertarget{xmlcmds_xmltag_seealso}%
\index{$<$seealso$>$@{$<$seealso$>$}} $<$seealso cref=\char`\"{}member\char`\"{}$>$}&Starts a \char`\"{}\+See also\char`\"{} section referring to \char`\"{}member\char`\"{}. Similar to using \mbox{\hyperlink{commands_cmdsa}{\textbackslash{}sa}} member. \\\cline{1-2}
{\ttfamily \label{xmlcmds_xmltag_summary}%
\Hypertarget{xmlcmds_xmltag_summary}%
\index{$<$summary$>$@{$<$summary$>$}} $<$summary$>$}&In case this tag is used outside a \mbox{\hyperlink{htmlcmds_htmltag_DETAILS}{$<$D\+E\+T\+A\+I\+LS$>$}} tag this tag identifies the brief description. Similar to using \mbox{\hyperlink{commands_cmdbrief}{\textbackslash{}brief}}. In case this tag is used inside a \mbox{\hyperlink{htmlcmds_htmltag_DETAILS}{$<$D\+E\+T\+A\+I\+LS$>$}} tag this tag identifies the heading of the \mbox{\hyperlink{htmlcmds_htmltag_DETAILS}{$<$D\+E\+T\+A\+I\+LS$>$}} tag. \\\cline{1-2}
{\ttfamily \label{xmlcmds_xmltag_term}%
\Hypertarget{xmlcmds_xmltag_term}%
\index{$<$term$>$@{$<$term$>$}} $<$term$>$}&Part of a \mbox{\hyperlink{xmlcmds_xmltag_list}{$<$list$>$}} command. \\\cline{1-2}
{\ttfamily \label{xmlcmds_xmltag_typeparam}%
\Hypertarget{xmlcmds_xmltag_typeparam}%
\index{$<$typeparam$>$@{$<$typeparam$>$}} $<$typeparam name=\char`\"{}param\+Name\char`\"{}$>$}&Marks a piece of text as the documentation for type parameter \char`\"{}param\+Name\char`\"{}. Similar to using \mbox{\hyperlink{commands_cmdparam}{\textbackslash{}param}}. \\\cline{1-2}
\label{xmlcmds_xmltag_typeparamref}%
\Hypertarget{xmlcmds_xmltag_typeparamref}%
\index{$<$typeparamref$>$@{$<$typeparamref$>$}}{\ttfamily $<$typeparamref name=\char`\"{}param\+Name\char`\"{}$>$}&Refers to a parameter with name \char`\"{}param\+Name\char`\"{}. Similar to using \mbox{\hyperlink{commands_cmda}{\textbackslash{}a}}. \\\cline{1-2}
{\ttfamily \label{xmlcmds_xmltag_value}%
\Hypertarget{xmlcmds_xmltag_value}%
\index{$<$value$>$@{$<$value$>$}} $<$value$>$}&Identifies a property. Ignored by doxygen. \\\cline{1-2}
{\ttfamily \label{xmlcmds_xmltag_CDATA}%
\Hypertarget{xmlcmds_xmltag_CDATA}%
\index{$<$"!\mbox{[}CDATA\mbox{[}...\mbox{]}\mbox{]}$>$@{$<$"![CDATA[...]]$>$}} $<$!\mbox{[}C\+D\+A\+TA\mbox{[}...\mbox{]}\mbox{]}$>$}&The text inside this tag (on the ...) is handled as normal doxygen comment except for the X\+ML special characters {\ttfamily $<$}, {\ttfamily $>$} and {\ttfamily \&} that are used as if they were escaped. \\\cline{1-2}
\end{longtabu}


Here is an example of a typical piece of code using some of the above commands\+:


\begin{DoxyCode}{0}
\DoxyCodeLine{\textcolor{comment}{/// <summary>}}
\DoxyCodeLine{\textcolor{comment}{}\textcolor{comment}{/// A search engine.}}
\DoxyCodeLine{\textcolor{comment}{}\textcolor{comment}{/// </summary>}}
\DoxyCodeLine{\textcolor{comment}{}\textcolor{keyword}{class }Engine}
\DoxyCodeLine{\{\textcolor{comment}{}}
\DoxyCodeLine{\textcolor{comment}{  /// <summary>}}
\DoxyCodeLine{\textcolor{comment}{  /// The Search method takes a series of parameters to specify the search criterion}}
\DoxyCodeLine{\textcolor{comment}{  /// and returns a dataset containing the result set.}}
\DoxyCodeLine{\textcolor{comment}{  /// </summary>}}
\DoxyCodeLine{\textcolor{comment}{  /// <param name="connectionString">the connection string to connect to the}}
\DoxyCodeLine{\textcolor{comment}{  /// database holding the content to search</param>}}
\DoxyCodeLine{\textcolor{comment}{  /// <param name="maxRows">The maximum number of rows to}}
\DoxyCodeLine{\textcolor{comment}{  /// return in the result set</param>}}
\DoxyCodeLine{\textcolor{comment}{  /// <param name="searchString">The text that we are searching for</param>}}
\DoxyCodeLine{\textcolor{comment}{  /// <returns>A DataSet instance containing the matching rows. It contains a maximum}}
\DoxyCodeLine{\textcolor{comment}{  /// number of rows specified by the maxRows parameter</returns>}}
\DoxyCodeLine{\textcolor{comment}{}  \textcolor{keyword}{public} DataSet Search(\textcolor{keywordtype}{string} connectionString, \textcolor{keywordtype}{int} maxRows, \textcolor{keywordtype}{int} searchString)}
\DoxyCodeLine{  \{}
\DoxyCodeLine{    DataSet ds = \textcolor{keyword}{new} DataSet();}
\DoxyCodeLine{    \textcolor{keywordflow}{return} ds;}
\DoxyCodeLine{  \}}
\DoxyCodeLine{\}}
\end{DoxyCode}


 